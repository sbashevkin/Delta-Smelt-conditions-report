\documentclass[]{article}
\usepackage{lmodern}
\usepackage{amssymb,amsmath}
\usepackage{ifxetex,ifluatex}
\usepackage{fixltx2e} % provides \textsubscript
\ifnum 0\ifxetex 1\fi\ifluatex 1\fi=0 % if pdftex
  \usepackage[T1]{fontenc}
  \usepackage[utf8]{inputenc}
\else % if luatex or xelatex
  \ifxetex
    \usepackage{mathspec}
  \else
    \usepackage{fontspec}
  \fi
  \defaultfontfeatures{Ligatures=TeX,Scale=MatchLowercase}
\fi
% use upquote if available, for straight quotes in verbatim environments
\IfFileExists{upquote.sty}{\usepackage{upquote}}{}
% use microtype if available
\IfFileExists{microtype.sty}{%
\usepackage{microtype}
\UseMicrotypeSet[protrusion]{basicmath} % disable protrusion for tt fonts
}{}
\usepackage[margin=1in]{geometry}
\usepackage{hyperref}
\hypersetup{unicode=true,
            pdftitle={Water conditions report},
            pdfborder={0 0 0},
            breaklinks=true}
\urlstyle{same}  % don't use monospace font for urls
\usepackage{graphicx,grffile}
\makeatletter
\def\maxwidth{\ifdim\Gin@nat@width>\linewidth\linewidth\else\Gin@nat@width\fi}
\def\maxheight{\ifdim\Gin@nat@height>\textheight\textheight\else\Gin@nat@height\fi}
\makeatother
% Scale images if necessary, so that they will not overflow the page
% margins by default, and it is still possible to overwrite the defaults
% using explicit options in \includegraphics[width, height, ...]{}
\setkeys{Gin}{width=\maxwidth,height=\maxheight,keepaspectratio}
\IfFileExists{parskip.sty}{%
\usepackage{parskip}
}{% else
\setlength{\parindent}{0pt}
\setlength{\parskip}{6pt plus 2pt minus 1pt}
}
\setlength{\emergencystretch}{3em}  % prevent overfull lines
\providecommand{\tightlist}{%
  \setlength{\itemsep}{0pt}\setlength{\parskip}{0pt}}
\setcounter{secnumdepth}{0}
% Redefines (sub)paragraphs to behave more like sections
\ifx\paragraph\undefined\else
\let\oldparagraph\paragraph
\renewcommand{\paragraph}[1]{\oldparagraph{#1}\mbox{}}
\fi
\ifx\subparagraph\undefined\else
\let\oldsubparagraph\subparagraph
\renewcommand{\subparagraph}[1]{\oldsubparagraph{#1}\mbox{}}
\fi

%%% Use protect on footnotes to avoid problems with footnotes in titles
\let\rmarkdownfootnote\footnote%
\def\footnote{\protect\rmarkdownfootnote}

%%% Change title format to be more compact
\usepackage{titling}

% Create subtitle command for use in maketitle
\providecommand{\subtitle}[1]{
  \posttitle{
    \begin{center}\large#1\end{center}
    }
}

\setlength{\droptitle}{-2em}

  \title{Water conditions report}
    \pretitle{\vspace{\droptitle}\centering\huge}
  \posttitle{\par}
    \author{}
    \preauthor{}\postauthor{}
    \date{}
    \predate{}\postdate{}
  

\begin{document}
\maketitle

\hypertarget{abiotic-drivers}{%
\section{Abiotic drivers}\label{abiotic-drivers}}

\begin{center}\includegraphics{Water-conditions-report_files/figure-latex/unnamed-chunk-1-1} \end{center}

Delta outflow is an estimate of the volumen of water entering San
Francisco Bay from the Delta. High Delta outflow is good for Delta Smelt
because\ldots{}

Outflow was average in 2018 but much lower than the record flows in
2017.

\begin{center}\includegraphics{Water-conditions-report_files/figure-latex/unnamed-chunk-2-1} \end{center}

X2 is a measure of salinity intrusion into the delta. It is defined as
the distance in kilometers from the golden gate where the salinity of
the bottom water reaches 2. Lower X2 is better for Delta smelt because
it means there is more low salinity habitat for them to occupy\ldots{}..

X2 was about average in 2018 but higher than the record low the previous
year.

\begin{center}\includegraphics{Water-conditions-report_files/figure-latex/unnamed-chunk-3-1} \end{center}

Delta smelt are sensitive to high water temperatures and their
sensitivity varies seasonally. More detail\ldots{}.

Temperatures were low throughout the Delta in 2018, except in the Upper
Sacramento.

\begin{center}\includegraphics{Water-conditions-report_files/figure-latex/unnamed-chunk-4-1} \end{center}

Delta smelt prefer fresh water and are most abundant in salinities of
1-2 ppt, are rare in salinities higher than 6 ppt, and are not found in
salinities above 14 ppt. A small percentage of Delta smelt are spawned
in the brackish waters, but use freshwater during winter and spring
months to spawn.

Salinity was average in 2018.

\begin{center}\includegraphics{Water-conditions-report_files/figure-latex/unnamed-chunk-5-1} \end{center}

Secchi depth is a measure of turbidity. Lower secchi depth indicates
higher turbidity, which is preferred by Delta Smelt. Delta Smelt evolved
in the turbid waters of the Delta and use the turbidity to hide from
predators and detect their prey or something.

Secchi depth was above average in most regions in the Delta in 2018,
reflecting a general trend of decreasing turbidity over the past few
years.

\hypertarget{biotic-drivers}{%
\section{Biotic drivers}\label{biotic-drivers}}

\begin{center}\includegraphics{Water-conditions-report_files/figure-latex/unnamed-chunk-6-1} \end{center}

Chlorophyll is a measure of primary productivity at the base of the
Delta food web. Higher chlorophyll indicates more food is available for
zooplankton which are important prey for many fish including Delta
Smelt. Chlorophyll levels above 10 μg/L are regarded as the minimum to
sustain good zooplankton growth and are shown in green.

Chlorophyll levels were low in 2018.

\begin{center}\includegraphics{Water-conditions-report_files/figure-latex/unnamed-chunk-7-1} \end{center}

Microcystis is a toxin-producing cyanobacteria harmful to human and
animal health. Microcystis is measured on a qualitative scale from 1
(none detected) to 5 (high concentrations).

High concentration microcystis blooms were detected in 2018 in the Lower
San Joaquin and Southern Delta regions.

\begin{center}\includegraphics{Water-conditions-report_files/figure-latex/unnamed-chunk-8-1} \end{center}

Phytoplankton are the base of the aquatic food web. They provide food
for zooplankton, which are important prey for many fishes such as Delta
Smelt.

Something about phytoplankton concentrations.

\begin{center}\includegraphics{Water-conditions-report_files/figure-latex/unnamed-chunk-9-1} \end{center}

Zooplankton are important food for Delta Smelt, which feed primarily on
these taxa

Zooplankton biomass was average in 2018 except in the Southern Delta and
Suisin regions where it was above average.

\begin{center}\includegraphics{Water-conditions-report_files/figure-latex/unnamed-chunk-10-1} \end{center}

Invasive bivalves (clams) have been responsible for drastic declines in
phytoplankton and zooplankton biomass in the Delta, reducing the amount
and quality of food available for fishes.

Bivalve abundances in 2018 were low or reduced relative to recent years
in the Lower San Joaquin, Southern Delta, and Suisin Bay/Marsh but very
high in the Lower Sacramento and Western Delta.

\hypertarget{delta-smelt}{%
\section{Delta Smelt}\label{delta-smelt}}

\begin{center}\includegraphics{Water-conditions-report_files/figure-latex/unnamed-chunk-11-1} \end{center}

Delta smelt abundance is measured by 4 IEP surveys that target different
life stages. Details here.

All abundance indices in 2018 were low or impossible to calculate due to
very low catch.


\end{document}
